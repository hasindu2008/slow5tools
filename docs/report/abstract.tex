\begin{abstract}
\label{sec:abstract}

Contemporary data storage of raw nanopore signals in the FAST5 file format doesn't benefit from parallel file access. A more computationally resourceful and space efficient file format could result in significant improvements in the runtime and storage size of nanopore sequencing pipelines.

    To address this issue, binary and compressed binary equivalents to the existing SLOW5 format were designed. Parallel access was implemented using SLOW5 index files and multithreading.

    Benchmarking experiments to determine the access time and file size of each SLOW5 format and their corresponding FAST5 files were performed using a sequenced human genome dataset on a rack-mounted server. The binary SLOW5 format was found to have the fastest access time with on average roughly 5000 reads accessed per second using 32 threads. Whilst the compressed binary SLOW5 file format was the most space efficient, using 250kB per read on average. In comparison, using the FAST5 format on average 100 reads could be accessed per second, with each read using roughly 300kB on average.

    By exploiting modern CPU\nomenclature{CPU}{Central processing unit} architectures with multithreading whilst employing space efficient compression techniques, the runtime and space requirements of nanopore sequencing pipelines can be significantly improved.

\end{abstract}
