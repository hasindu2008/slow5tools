\section{Discussion}
\label{sec:disc}

\subsection{Evaluation}
\label{sec:disc:eval}

The results displayed in section \ref{sec:experi:results} show how storing ONT-sequenced reads in the SLOW5 file format can significantly improve the runtime of nanopore-specific pipelines. In particular, the binary encoded SLOW5 reads could be accessed roughly 44 times faster than FAST5-stored reads. For a small human genome sequencing dataset, this is the difference between waiting 1.36 minutes and 1 hour to retrieve 8.7 gigabases worth of reads. For the binary compressed encoding this is the equivalent of waiting almost 2 minutes, which is only about 1.4 times longer than it takes for the binary encoding. The regular ASCII encoding would take 2.73 minutes.

The differences in access speed between the SLOW5 encodings can be explained by the size of each read and the process of decoding each read into the ASCII format. The original ASCII-encoded SLOW5 format takes up roughly 2 and 3.4 times more space than the binary and binary compressed encodings respectively. This translates into higher I/O costs for reading the ASCII encoding. In comparison the binary compressed encoding suffers fewer I/O costs but each read must be decompressed and converted from binary into the ASCII format. This memory-decoding trade-off seems to favour decoding as the number of threads increase, as is evident in Figure \ref{fig:time} as the binary compressed encoding (in pink) overlaps the ASCII encoding (in orange) between 8 and 16 threads. This suggests that decoding is an overhead that's overcome by more parallel threads of execution.

For the typical nanopore sequencing pipeline, the results suggest that the compressed binary encoding of the SLOW5 file format is most appropriate, with a 20\% reduction in file size and 31 times increase in read access performance. However, for maximum performance the compressed binary encoding could be used for long-term storage and then decompressed pre-emptively whenever read access is required.

For adoption into the nanopore sequencing community, the content and structure of the SLOW5 file format should first of all be finalised. Then, a simple toolkit for lossless conversion between FAST5 and the SLOW5 encodings, reading, sorting, indexing and other operations should be built. This should take the form of a command line interface (CLI\nomenclature{CLI}{Command-line interface}) and libraries for common bioinformatics programming languages. Afterwards, SLOW5 should be integration into popular third party software tools (e.g. Bonito \cite{bonito}, Nanopolish \cite{nanopolish} and f5c \cite{f5c}) and released as an open source resource for the bioinformatics community.

\subsection{Limitations}
\label{sec:disc:limit}

The design of the experiments could be improved on with more datasets capturing a diverse range of organisms each with multiple lists of read IDs. Furthermore, different servers with various configurations such as a RAID\nomenclature{RAID}{Redundant Array of Inexpensive Disks} data storage system, should have been used to generate more robust and generalisable conclusions.
